\section{Appendix A: Mathematical Definitions}
\label{app:algebra}

\subsection{Cayley-Dickson Construction}
The Sedenion algebra $\mathbb{S}$ is constructed from the Octonions $\mathbb{O}$ via the Cayley-Dickson doubling process. Let $a, b \in \mathbb{O}$. A sedenion is defined as a pair $(a, b)$. The multiplication rule is:
\begin{equation}
(a, b)(c, d) = (ac - d^*b, da + bc^*)
\end{equation}
where $^*$ denotes conjugation. This process generates a 16-dimensional non-associative algebra.

\subsection{G$_2$ Automorphisms}
The group G$_2$ is the automorphism group of the Octonions. It acts on the imaginary basis $e_1, \dots, e_7$ preserving the multiplication structure. The generators of G$_2$ can be explicitly represented as linear combinations of the derivations $D_{ij} = [L_{e_i}, L_{e_j}] + [R_{e_i}, R_{e_j}]$.

\subsection{Associator and Non-Associativity}
The non-associativity of the algebra is quantified by the Associator:
\begin{equation}
[x, y, z] = (xy)z - x(yz)
\end{equation}
In Sedenions, this term is non-zero and generates the mass gap in the gauge theory.
