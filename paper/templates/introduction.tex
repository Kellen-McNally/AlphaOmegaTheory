We present a framework based on a single geometric principle:

\begin{center}
\textit{The fundamental substrate of reality is a Unified Sedenion Field transforming under G$_2$ geometry.}
\end{center}

The sedenion is a 16-dimensional hypercomplex number system that naturally splits into
\textbf{two 8-dimensional octonions}:
\begin{align}
\text{External octonion } \mathbb{O}_{ext} &: \quad \{e_0, e_1, e_2, e_3, e_4, e_5, e_6, e_7\} \quad \text{(spacetime + phase space)} \\
\text{Internal octonion } \mathbb{O}_{int} &: \quad \{i_0, i_1, i_2, i_3, i_4, i_5, i_6, i_7\} \quad \text{(particle physics)}
\end{align}

Each 8-dimensional octonion has the exceptional Lie group G$_2$ as its automorphism group.
This G$_2$ structure is characterized by exactly \textbf{two numbers}:
\begin{equation}
\boxed{\tau = {{tau}} \quad \text{and} \quad \dim(G_2) = {{dim}}}
\end{equation}

From these two mathematical facts—$\tau = {{tau}}$ (triality) and $\dim(G_2) = {{dim}}$—emerge all predictions:
\begin{itemize}
\item Three fermion generations (from $\tau = {{tau}}$)
\item Gauge coupling unification at $\alpha_{GUT} = 1/(\tau \times {{dim}}) = {{alpha_gut:.10f}}$
\item Dark energy density $\Omega_\Lambda = {{c3}}/{{dim_plus_rank}} = {{omega_lambda:.6f}}$ (matching Planck observations)
\item Weak mixing angle $\sin^2\theta_W = {{tau}}/{{dim_minus_1}} = {{sin2_theta_w:.6f}}$ (matching precision measurements)
\item All fermion masses, mixing angles, and cosmological parameters
\end{itemize}

Every prediction is fully constrained by G$_2$ structure—over 50 independent observables traced
to two mathematical facts.

\subsection{Motivation}

The Standard Model requires 26 free parameters (12 fermion masses, 3 coupling constants, 2 Higgs parameters, 8 mixing angles/phases, and 1 strong CP phase) and cannot explain the cosmological constant,
dark matter, three generations, or the strong CP angle. This framework provides a \textbf{geometric derivation of the mass scales and couplings}
from pure G$_2$ geometry, following the principle that physical constants should emerge from
mathematical structure.

\subsection{Roadmap}

The paper is organized into five logical parts:

\textbf{Part I: Foundations} (Sec. 1-5) establishes the geometric axioms, the "Zero-Parameter" methodology, the sedenion manifold structure, and computational verification of the algebra.

\textbf{Part II: Core Predictions} (Sec. 6-15) derives the fundamental constants of particle physics, including the Grand Unified coupling, Weak mixing angle, Dark Energy, Dark Matter (Temporal Antimatter), neutrino masses, and the full fermion mass spectrum.

\textbf{Part III: Unified Theory} (Sec. 16-20) presents the Complete Action, the Sedenion Dirac equation, proton decay lifetime, and the unification of spacetime and internal quantum states via sedenion geometry.

\textbf{Part IV: Cosmology \& Gravity} (Sec. 21-25) derives General Relativity as an algebraic restoring force, solves the Hierarchy Problem, explains atomic structure stability, and details the S-curve thermal evolution of the universe.

\textbf{Part V: Advanced Implications} (Sec. 26-28) explores the resolution of singularities (Planck Stars), the algebraic origin of quantum logic and entanglement, and the number-theoretic properties of the predictions.

\subsection{Empirical Foundation}

The Standard Model successfully describes three fundamental forces but requires 26 free parameters (12 fermion masses, 3 coupling constants, 2 Higgs parameters, 8 mixing angles/phases, and 1 strong CP phase). Deriving physical constants from pure mathematics has been a central goal since
Einstein's unified field theory program.

Einstein famously stated that ``there are no arbitrary constants'' in a final theory \cite{Einstein1945}, arguing that nature's laws should be logically determined. Similarly, Paul Dirac argued that large dimensionless numbers in physics must be interconnected \cite{Dirac1937}. More recently, physicists like Lee Smolin have criticized the reliance on parameter tuning \cite{Smolin2006}, and Alexander Unzicker has called for a return to natural philosophy where constants are derived rather than measured \cite{Unzicker2010}.

We present an empirically validated framework based on G$_2$ Lie algebra geometry that
derives all dimensionless physical constants from two mathematical properties: the triality order
$\tau = {{tau}}$ and the dimension $\dim(G_2) = {{dim}}$. Dimensional constants serve only as conversion factors between human-chosen units and natural geometric scales: $c$ converts time to space, $\hbar$ converts frequency to energy, $G$ converts mass to curvature, and $e$ converts geometry to electromagnetic coupling. All
predictions arise from the intrinsic structure of G$_2$ as classified by Cartan.
Following this historical mandate, we demonstrate that physical observables emerge from pure
geometry, achieving {{ global_agreement_fmt }}\% agreement with experiment.

\subsection{Historical Context: Cartan's Classification}

In 1894, Élie Cartan \cite{Cartan1894} completed the classification of simple Lie algebras begun
by Wilhelm Killing \cite{Killing1888}. Beyond the four infinite
families (A$_n$, B$_n$, C$_n$, D$_n$), Cartan identified five exceptional cases.
The smallest is G$_2$, characterized by:
\begin{align}
    \text{rank}(G_2) &= {{rank}} \\
    \dim(G_2) &= {{dim}} \quad \text{(adjoint representation)} \\
    \tau &= {{tau}} \quad \text{(triality order, } \tau^3 = 1\text{)}
\end{align}

These numbers are not adjustable—they are the defining properties of G$_2$, fixed
by representation theory. The dimension {{dim}} counts the generators of the Lie
algebra (gauge bosons in physics). The triality $\tau = {{tau}}$ is the order of
the unique outer automorphism, a feature not shared by other exceptional groups.

\subsection{Why G$_2$? Mathematical Uniqueness and Empirical Validation}

We select G$_2$ based on two independent criteria: mathematical uniqueness and
empirical validation.

\paragraph{Smallest Exceptional Group}
G$_2$ is the smallest of the five exceptional Lie algebras (G$_2$, F$_4$, E$_6$,
E$_7$, E$_8$). With dimension {{dim}}, it provides the minimal exceptional structure:
\begin{align}
\dim(G_2) &= {{dim}} \\
\dim(F_4) &= 52 \\
\dim(E_6) &= 78 \\
\dim(E_7) &= 133 \\
\dim(E_8) &= 248
\end{align}

This minimality is crucial—smaller dimension means fewer generators, imposing
maximal constraint with minimal structure. By Occam's razor, we explore the smallest
exceptional group first.

\paragraph{Unique 7-Dimensional Structure}
Joyce \cite{Joyce1996} proved G$_2$ is the unique holonomy group for Ricci-flat 7-manifolds.
G$_2$ is the automorphism group of 7D imaginary octonions, preserves a unique associative
3-form, and corresponds to the 7D spinor representation.

\paragraph{Automorphism Group of Octonions}

G$_2$ is the automorphism group of octonions: $G_2 = \text{Aut}(\mathbb{O})$.
By Hurwitz (1898) \cite{Hurwitz1898}, there are exactly four normed division algebras:
$\mathbb{R}$ (dim 1), $\mathbb{C}$ (dim 2), $\mathbb{H}$ (dim 4), $\mathbb{O}$ (dim 8).
Octonions are maximal; their automorphism group G$_2$ represents the symmetry of the largest
normed division algebra. Furey (2016, 2018) \cite{Furey2016, Furey2018} showed Standard Model structure
emerges from octonionic algebra.

\paragraph{Unique Triality Structure}
The **Octonion algebra** possesses a unique automorphism of order 3 (triality), which is preserved by its automorphism group G$_2$.
This unique property directly explains the three-generation structure of fermions,
a feature with no explanation in the Standard Model.

\paragraph{Synthesis}
These four independent mathematical arguments—minimality, Joyce's holonomy theorem,
maximal division algebra, and unique triality—all point to G$_2$ as the geometrically
natural choice for a unified framework. The convergence of these mathematical facts
suggests G$_2$ is not merely one possibility among many, but the structure demanded
by consistency requirements for a theory with seven compact dimensions.

\subsection{The Two Fundamental Numbers}

All predictions in this framework derive from combining $\tau = {{tau}}$ and $\dim(G_2) = {{dim}}$
with the cubic Casimir $C_3(G_2) = {{c3}}$ and rank {{rank}}. These yield:

\paragraph{Grand Unified Coupling}
{\small
\[\alpha_{GUT} = \frac{1}{\tau \times \dim(G_2)} = \frac{1}{ {{ tau }} \times {{ dim }} } = \frac{1}{ {{ tau_times_dim }} } = {{ alpha_gut:.10f }}\]
}

The number {{ tau_times_dim }} = ${{ tau }} \times {{ dim }}$ appears as the inverse of the gauge coupling at the
GUT scale. This agrees with renormalization group evolution
from measured low-energy couplings to sub-percent accuracy.

\paragraph{Weak Mixing Angle}
{\small
\[\sin^2\theta_W = \frac{\tau}{\dim(G_2) - 1} = \frac{ {{ tau }} }{ {{ dim }} - 1} = \frac{ {{ tau }} }{ {{ dim_minus_1 }} } = {{ sin2_theta_w:.6f }}\]
}

The Weinberg angle characterizes electroweak symmetry breaking. Experiment yields:
\[\sin^2\theta_W(M_Z) = {{ sin2_theta_w_exp:.5f }} \pm {{ sin2_theta_w_err:.5f }}\]
which agrees to {{ sin2_error:.2f }}\%. 

\paragraph{Dark Energy Density}
{\small
\[\Omega_\Lambda = \frac{C_3(G_2)}{\dim(G_2) + \text{rank}(G_2)} = \frac{ {{ c3 }} }{ {{ dim }} + {{ rank }} } = \frac{ {{ c3 }} }{ {{ dim_plus_rank }} } = {{ omega_lambda:.6f }}\]
}

The cosmological constant density parameter. Planck 2018 \cite{Planck2018} observes
$\Omega_\Lambda = {{omega_lambda_exp:.5f}} \pm {{omega_lambda_err:.5f}}$,
with {{omega_error:.2f}}\% discrepancy from
the geometric prediction.
\paragraph{Three Generations}

\[N_{gen} = \tau = {{tau}}\]

The triality order directly predicts three fermion generations. LEP measurements \cite{LEP1996}
of the Z boson width confirm $N_{gen} = 2.9840 \pm 0.0082$ light neutrino species,
consistent with exactly {{tau}}.

\paragraph{Strong CP Solution}

\[\theta_{QCD} = 0 \quad \text{(exact)}\]

The Z$_3$ discrete symmetry from triality constrains the QCD vacuum angle to
$\theta \in \{0, 2\pi/3, 4\pi/3\}$. CP conservation selects $\theta = 0$,
solving the strong CP problem without axions.

All numerical values are computed procedurally from the G$_2$ structure constants.
The only inputs are $\tau = {{tau}}$ and $\dim(G_2) = {{dim}}$, which are mathematical
facts about G$_2$.