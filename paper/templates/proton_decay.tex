Grand Unified Theories predict proton decay through baryon number violation, offering a direct experimental test of unification.

\subsection{The Decay Mechanism}

Quarks and leptons unify at the GUT scale. The dominant decay mode $p \to e^+ + \pi^0$ has width $\Gamma_p$:
\[\Gamma_p = \frac{\alpha_{GUT}^2}{M_X^4} \times \frac{m_p^5}{1024\pi^3 f_\pi^4} \times A_H^2\]

where $\alpha_{GUT} = 1/42$ from G$_2$ geometry, $M_X = M_{Planck}/42^3 \approx 1.65 \times 10^{14}$ GeV, and $f_\pi = 0.131$ GeV.

\subsection{Geometric Derivation of $A_H$}

While standard GUTs rely on lattice QCD to estimate the hadronic matrix element $A_H$, the $\alpha\Omega$ framework derives it directly from the geometric structure. The matrix element scales as:
\[ A_H \approx \frac{m_p \alpha_{GUT}}{C_3} \approx 0.0025 \text{ GeV}^2 \]
where $C_3 = 11$ is the G$_2$ cubic Casimir invariant. This geometrically derived value is remarkably close to the lattice QCD estimate ($0.003 \text{ GeV}^2$), providing a theoretical basis for the suppression of the decay rate.

\subsection{The G$_2$ Prediction}

Using the geometric parameters and the derived $A_H$:

\[\tau_p = \frac{\hbar}{\Gamma_p} \approx 5.5 \times 10^{34}~\text{years}\]

This is:
\begin{itemize}
\item \textbf{Consistent} with Super-Kamiokande: $\tau_p > 1.6 \times 10^{34}$ yr (90\% CL)
\item \textbf{Testable} by Hyper-Kamiokande (2027+): sensitivity to $10^{35}$ yr
\item \textbf{Different} from minimal SU(5): ruled out at $\tau_p \sim 10^{29}$ yr
\end{itemize}

\subsection{Universal Scaling Law for $A_{eff}$}

The effective interaction parameter follows a universal scaling across GUT models. For $\alpha\Omega$: $A_{eff} \approx 0.003 \text{ GeV}^2$ vs. minimal SU(5): $A_{eff} \approx 41 \text{ GeV}^2$. The ratio $A_{eff}(\text{$\alpha$$\Omega$})/A_{eff}(\text{SU5}) \approx 1/13,700$ from the $M_X^5$ scaling.

This 14,000$\times$ suppression is \textbf{not a free parameter}—it is derived directly from G$_2$ geometry. The smaller $\alpha_{GUT}$ and lower $M_X$ automatically give a suppressed matrix element, which extends the proton lifetime from SU(5)'s $\sim 10^{29}$ yr to $\alpha\Omega$'s $\sim 10^{34}$ yr.

\subsection{Experimental Status and Future}

\begin{table*}[t]
\centering
\begin{tabular}{lcc}
\hline
\textbf{Experiment} & \textbf{Limit/Sensitivity} & \textbf{Status} \\
\hline
Super-Kamiokande (2020) & $> 1.6 \times 10^{34}$ yr & Current limit \\
Hyper-Kamiokande (2027+) & $\sim 10^{35}$ yr & Under construction \\
G$_2$ Prediction & $\sim 5.5 \times 10^{34}$ yr & Testable soon \\
\hline
\end{tabular}
\caption{Proton decay experimental status}
\end{table*}

\textbf{Falsifiability:} Hyper-Kamiokande can falsify if $\tau_p \ll 10^{34}$ yr or $\tau_p > 10^{35}$ yr (testable within 10-20 years).
