Having established the geometric prediction of dark matter fraction
(Section \ref{sec:temporal_antimatter}), we investigate the stability
of the Casimir operators over cosmic time.

\subsection{Casimir Operators: Invariants vs. Effective Parameters}

In standard Lie algebra representation theory, Casimir operators are
\textbf{invariants}. However, in a \textbf{dynamical cosmological context},
one might ask if the effective Casimir value evolves.

\begin{figure}[h]
\centering
\includegraphics[width=0.8\textwidth]{figures/totani_resonance.pdf}
\caption{Casimir Stability: The symmetric fixed point at $C_3=11$ prevents runaway instabilities.}
\label{fig:totani_resonance}
\end{figure}

From the circular geometry:
\begin{equation}
 f_{\text{dark}} = \frac{C_3}{13}
\end{equation}

Testing this against cosmic epochs reveals that $C_3 \approx 11$ is consistent
with the present era ($f_{\text{dark}} \approx {{ omega_dm_pct_1f }}\%$).

\subsection{The Symmetric Fixed Point}

Why does the universe select $C_3 = 11$?
The total cubic Casimir capacity of the Sedenion ($\mathbb{O}_{ext} \oplus \mathbb{O}_{int}$) is:
\begin{equation}
 C_3^{\text{total}} = C_3^{\text{ext}} + C_3^{\text{int}} = 11 + 11 = 22
\end{equation}

The observed value $C_3 = 11$ represents exactly \textbf{half} of the total algebraic capacity:
\begin{equation}
 \frac{C_3^{\text{obs}}}{C_3^{\text{total}}} = \frac{11}{22} = 50\%
\end{equation}

This suggests that the universe resides in a \textbf{symmetric fixed point} where the
topological load is shared equally between the vacuum potential and the latent geometry.
This symmetry argues for the stability of $C_3$ over time, supporting the hypothesis of a Constant Casimir.

\subsection{Implications for Cosmic Fate}

With $C_3$ fixed at 11, the dark matter fraction remains constant at $11/13 \approx {{ omega_dm_pct_1f }}\%$.
This implies the universe will not undergo a "Big Rip" (phantom energy) or a "Big Crunch"
(if geometry were to collapse). Instead, the algebraic constraints ensure a stable
asymptotic evolution toward the thermal equilibrium described in Section \ref{sec:cosmological_expansion_s_curve_evolution}.

\subsection{Summary}

The Sedenion algebraic structure provides a rigid scaffold ($C_3=11$, rank=2) that
constrains the evolution of the universe, preventing runaway instabilities and
fixing the dark sector ratios.
