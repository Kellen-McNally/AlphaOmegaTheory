The $\alpha \Omega$ framework predicts that cosmological expansion follows an S-curve toward thermodynamic equilibrium, fundamentally different from the asymptotic de Sitter expansion of standard $\Lambda$CDM cosmology.

\subsection{S-Curve Expansion from Thermodynamics}

Just as the matter/dark-matter ratio evolves via logistic S-curve toward equilibrium, the expansion rate itself must follow the same pattern due to thermodynamic constraints.

\begin{figure}[ht]
\centering
\includegraphics[width=0.8\textwidth]{figures/scurve_expansion.pdf}
\caption{Comparison of Cosmic Evolution: The $\alpha\Omega$ S-curve (red) vs Standard $\Lambda$CDM (blue). The models agree during the matter-dominated era but diverge in the future: $\Lambda$CDM accelerates exponentially (Heat Death), while the Sedenion Universe relaxes into a stable thermal equilibrium.}
\label{fig:scurve}
\end{figure}

\textbf{Critical clarification}: The S-curve describes the \emph{expansion rate} $H(t)$, not the total size $a(t)$. The universe continues expanding \emph{forever}, but the rate of expansion approaches a constant equilibrium value rather than accelerating exponentially. This is \emph{not} heat death—the equilibrium state maintains a finite temperature $T_{eq} \neq 0$ indefinitely.

The expansion history exhibits three distinct phases:
\begin{enumerate}
\item \textbf{Deceleration Phase} ($t < 9$ Gyr): Matter-dominated era
\item \textbf{Acceleration Phase} ($9 < t < 22$ Gyr): Dark energy dominates
  We are currently at $t = 13.8$ Gyr, 85\% through this phase.
\item \textbf{Equilibrium Phase} ($t > 22$ Gyr): Thermodynamic balance
  This state represents the de Sitter vacuum temperature corresponding to the remnant dark energy density.
\end{enumerate}

\subsection{Matter/Dark-Matter Ratio Evolution}

The matter-to-dark-matter ratio follows a logistic S-curve.
Current observations:
\begin{itemize}
\item CMB ($t = 0.38$ Gyr): $r \approx 0.95$ (nearly all baryonic matter)
\item Transition ($t = 9$ Gyr): $r \approx 0.50$ (equal matter/dark-energy)
\item Present ($t = 13.8$ Gyr): $r = 0.185 \pm 0.010$ (85\% toward equilibrium)
\end{itemize}

The excellent agreement between the fit ($r = 0.185$) and observation ($r_{obs} = 0.156/0.844 = 0.185$) validates the S-curve model.

\subsection{Comparison with Standard $\Lambda$CDM}

Standard $\Lambda$CDM predicts asymptotic exponential expansion, leading to 'heat death': $T \to 0$.
The $\alpha \Omega$ framework predicts S-curve equilibrium leading to a \textbf{thermal bath}: finite equilibrium temperature, linear expansion.

The S-curve model predicts a younger universe ($8.83$ Gyr) compared to $\Lambda$CDM ($13.8$ Gyr) for the same redshift data. This discrepancy is a falsifiable prediction of the model, implying a different expansion history $H(t)$ during the early matter-dominated era.

% Section removed to maintain consistency with time-invariant geometric derivation.