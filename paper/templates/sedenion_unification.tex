
The unification of quantum mechanics and general relativity has been the "Holy Grail" of physics for a century. The $\alpha\Omega$ framework proposes that this unification occurs not by quantizing gravity, but by recognizing that both are projections of a single 16-dimensional algebraic structure: the Sedenion ($\mathbb{S}$). 

\subsection{The Algebraic Split}

The sedenion algebra is non-associative, but it naturally decomposes into two associative octonionic subalgebras that are coupled via the Cayley-Dickson process. We identify these as:

\begin{equation}
\mathbb{S} = \mathbb{O}_{ext} \oplus \mathbb{O}_{int}
\end{equation}

\begin{enumerate}
\item \textbf{External Octonion ($\mathbb{O}_{ext}$)}: This 8D space represents the "stage" of physics—spacetime and phase space. Its basis elements {e$_0$, …, e$_7$} correspond to time, energy, position, and momentum. This sector is governed by classical logic and General Relativity.
\item \textbf{Internal Octonion ($\mathbb{O}_{int}$)}: This 8D space represents the "actors" on the stage—the quantum states of particles. Its basis elements {i$_0$, …, i$_7$} correspond to the quantum numbers (charge, spin, flavor) of the Standard Model. This sector is governed by quantum logic and Unitary Evolution.
\end{enumerate}

The unification is not a blending of the two, but a \textbf{dual aspect} reality.

\subsection{The Measurement Problem}

One of the deepest puzzles in quantum mechanics is the "Measurement Problem": how does a probabilistic quantum wavefunction collapse into a definite classical reality upon observation?

In the Sedenion framework, measurement is a \textbf{geometric interaction} between the Internal and External octonions.

\subsubsection{The Algebra of Observation}

A "particle" is an excitation in $\mathbb{O}_{int}$. An "observer" (or apparatus) exists in $\mathbb{O}_{ext}$. An interaction between them is a sedenion multiplication:
\[S_{state} = O_{ext} \times O_{int}\]

Because the sedenion algebra is \textbf{non-associative}, the order of operations matters:
\[(A \times B) \times C \neq A \times (B \times C)\]

This non-associativity creates a "selection mechanism." When the internal quantum state interacts with the macroscopic external geometry, the algebra forces a choice of associativity path. This choice corresponds to the "collapse" of the wavefunction.

\begin{itemize}
\item \textbf{Coherent Evolution}: As long as the system remains within $\mathbb{O}_{int}$ (isolated quantum system), associativity holds (octonions are associative), and unitary evolution preserves superposition.
\item \textbf{Measurement}: When the system couples strongly to $\mathbb{O}_{ext}$ (measurement), the combined system enters the non-associative 16D regime. The "ambiguity" of the product resolves into a specific classical outcome.
\end{itemize}

The "collapse" is not a random, mystical process but a \textbf{algebraic phase transition} from associative to non-associative geometry.

\subsection{Complete Physical State}

The full state of the universe is described by the 16-component Sedenion wavefunction $\Psi \in \mathbb{S}$.
\[\Psi(x) = \psi_{ext}(x) + \psi_{int}(x) \cdot E\]
where $E$ is the sedenion unit connecting the two sectors.

The G$_2$ automorphism acts independently on each octonion, preserving the respective structures. In the $\alpha\Omega$ framework, we \textbf{impose a diagonal $G_2$ symmetry} across both sectors, ensuring that the symmetries of spacetime (Lorentz) and internal states (Gauge) are compatible within the non-associative sedenion product.

\subsection{Gravity and Gauge Forces}

\begin{itemize}
\item \textbf{Gravity} arises from the curvature of the $\mathbb{O}_{ext}$ manifold. It couples to energy-momentum ($T_{\mu\nu}$), which is the norm of the sedenion state. Since the norm is multiplicative even in non-associative algebras ($\|xy\| = \|x\|\|y\|$), gravity sees the "whole picture"—it couples to both classical and quantum energy.
\item \textbf{Gauge Forces} arise from the curvature of the $\mathbb{O}_{int}$ manifold. They couple to internal quantum numbers. Because $\mathbb{O}_{int}$ is orthogonal to $\mathbb{O}_{ext}$, gauge forces are confined to the "internal" space—we see them as charges, not spatial directions.
\end{itemize}

This geometric separation explains why we perceive 4 dimensions of space-time, while the other dimensions manifest as forces.
