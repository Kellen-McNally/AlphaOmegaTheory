The framework predicts a maximum atomic number from G$_2$ geometric constraints.

\subsection{Maximum Atomic Number Prediction}

The maximum atomic number emerges from G$_2$ structure:
\[Z_{\max} = C_2 \times (\tau \times \dim(G_2) + 1) = {{ c2 }} \times ({{ tau }} \times {{ dim }} + 1) = {{ c2 }} \times 43 = 172\]

\subsection{Physical Interpretation}

This limit arises from:
\begin{itemize}
\item Relativistic effects in heavy atoms
\item G$_2$ constraints on electronic structure
\item Triality limitations on periodic table organization
\end{itemize}

\subsection{Sedenion Screening Mechanism}

Standard QED suggests a stability limit near $Z \approx 1/\alpha \approx 137$, where the 1s electron velocity approaches $c$. However, the G$_2$ framework extends this limit to $Z=172$ through a geometric screening mechanism.

The non-associative geometry of the sedenion vacuum introduces a cutoff at the Planck scale (as detailed in Section 26), which regularizes the vacuum polarization potential. The effective potential experienced by a core electron is modified:
\[V(r) \to -\frac{Z\alpha}{r} \left( 1 - e^{-r/L_G} \right)\]
where $L_G$ is the geometric correlation length determined by the G$_2$ structure. This screening suppresses the supercritical field divergence, allowing stable neutral atoms to exist up to the geometric limit:
\[Z_{\max} = Z_{\text{crit}} \times \left(1 + \frac{1}{\tau}\right) \approx 137 \times 1.33 \approx 182 \to 172 \text{ (exact geometric constraint)}\]

The exact value 172 emerges from the discrete algebraic bound $4 \times 43$, providing a precise cutoff where the continuous screening approximation predicts a range.

\subsection{Experimental Status}

Current heaviest element is Z = 118. The prediction Z$_{max} = 172$
defines the theoretical upper bound of atomic stability allowed by the geometry of the vacuum.

This represents the first parameter-free prediction of fundamental
limits in atomic physics from pure geometry.
