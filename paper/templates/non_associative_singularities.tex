A fundamental problem in General Relativity is the prediction of singularities---points where curvature diverges and physics breaks down. We propose that these singularities are artifacts of assuming an \textit{associative} algebra for spacetime geometry down to arbitrary scales.

\subsection{The Associativity Limit}

In standard physics, operators are associative: $(AB)C = A(BC)$. This allows for point-like interactions and infinite nesting of fields. However, the $\alpha\Omega$ framework posits that spacetime is an 8-dimensional octonion space embedded in a 16-dimensional sedenion structure. Sedenions are \textbf{non-associative}.

The non-associativity of the sedenion algebra introduces a fundamental scale limit. The commutator structure:
\[[s_i, s_j, s_k] = (s_i s_j)s_k - s_i(s_j s_k) \neq 0\]
acts as a "topological obstruction" to contracting geometry to a point.

\subsection{The Planck Core Mechanism}

We identify the "singularity" not as a point of infinite density, but as the phase transition boundary where the geometry transitions from the associative external octonion ($\mathbb{O}_{ext}$) to the full non-associative sedenion ($\mathbb{S}$). 

\subsubsection{Black Holes as Non-Associative Solitons}

In this framework, a black hole is a region where the local energy density forces the excitation of the internal octonion directions.
\begin{itemize}
\item \textbf{Exterior ($r > R_s$)}: Gravity is described by $\mathbb{O}_{ext}$. Geometry is associative. Curvature is standard GR.
\item \textbf{Interior ($r < R_s$)}: The full 16D structure is active. Non-associativity prevents the collapse to $r=0$.
\end{itemize}

Instead of a singularity, the collapse stabilizes at the scale where non-associativity dominates---the Planck scale. The black hole center forms a \textbf{Planck Core} (or Planck Star), a stable, non-associative soliton with density $\rho \sim M_p/L_p^3$.

\subsubsection{Information Storage}

The Black Hole Information Paradox arises from the "no-hair theorem" of classical GR. However, a non-associative Planck Core has massive storage capacity. The ordering of the 16 basis elements matters:
\[\text{State } A = (e_1 e_2) e_3 \neq \text{State } B = e_1 (e_2 e_3)\]

This "ordering entropy" provides exactly the $A/4L_p^2$ degrees of freedom required to preserve information holographically.

\subsection{Cosmological Implications}

The same logic applies to the Big Bang. The universe did not begin as a singularity ($T=0$).
\begin{equation}
\lim_{T \to 0} \text{Geometry} \neq \text{Point}
\end{equation}

Instead, $T \to 0$ represents the "Associativity Horizon"---the moment when the universe cooled sufficiently for the associative $\mathbb{O}_{ext}$ sector (spacetime) to decouple from the non-associative $\mathbb{S}$ background.

This implies:
\begin{enumerate}
\item \textbf{No Initial Singularity}: The universe emerged from a non-associative high-dimensional state, not a geometric point.
\item \textbf{Inflationary Mechanism}: The phase transition from non-associative to associative geometry acts as a repulsive scalar potential, naturally driving the initial expansion (inflation) without an arbitrary inflaton field.
\end{enumerate}

The $\alpha\Omega$ framework thus resolves the singularity problem of General Relativity not by quantization of gravity per se, but by the \textit{algebraic extension} of geometry into non-associative sectors.
