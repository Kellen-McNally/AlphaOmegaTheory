The Type I Seesaw Mechanism is the standard explanation for the smallness of neutrino masses. It posits that light neutrinos are Majorana particles, their mass generated by the suppression of heavy right-handed neutrinos.

\subsection{The Seesaw Formula}

The neutrino mass matrix in the basis $(\nu_L, N_R)$ is given by:
\begin{equation}
M_\nu = \begin{pmatrix} 0 & M_D \\ M_D^T & M_R \end{pmatrix}
\end{equation}

where $M_D$ is the Dirac mass matrix (order of the charged lepton/quark masses) and $M_R$ is the heavy Majorana mass matrix for the right-handed neutrinos.

Diagonalizing this matrix yields the light neutrino masses:
\begin{equation}
m_\nu \approx -M_D M_R^{-1} M_D^T
\end{equation}

Assuming the matrices are diagonal in the mass basis:
\begin{equation}
m_{\nu, i} \approx \frac{Y_{\nu, i}^2 v^2}{M_{R, i}}
\end{equation}

where $v \approx 246$ GeV is the Higgs VEV and $Y_\nu$ are the Yukawa couplings.

\subsection{Geometric Hierarchy from G$_2$}

In standard GUTs, $M_R$ and $Y_\nu$ are free parameters. In the $\alpha\Omega$ framework, they are fixed by G$_2$ geometry.

\subsubsection{Right-Handed Mass Scale}

The overall scale for the right-handed neutrinos is the GUT scale, modified by the geometric factor $\tau^2/7$:
\begin{equation}
M_{R, 3} = M_{GUT} \times \frac{\tau^2}{\dim(\mathbf{7})} = M_{GUT} \times \frac{9}{7}
\end{equation}

This factor $9/7$ arises from the ratio of the generation manifold dimension ($\tau^2=9$) to the fundamental representation dimension ($7$). 

\subsubsection{Generational Structure}

The hierarchy between the three right-handed neutrinos is determined by the "Triality Cascade":
\begin{itemize}
\item \textbf{Third Generation (Heaviest)}: $M_{R,3}$ couples to the full 9-dimensional generation space.
\item \textbf{Second Generation}: $M_{R,2}$ couples to the 8-dimensional adjoint subspace (spinors). Ratio: $M_{R,2}/M_{R,3} = 7/8$ (normalized to spinor dimension).
\item \textbf{First Generation (Lightest)}: $M_{R,1}$ couples to the 1-dimensional singlet space (the triality invariant). Ratio: $M_{R,1}/M_{R,3} = 1/20$ (from the dimension of the antisymmetric tensor representation $\mathbf{20}$ in the embedding).
\end{itemize}

This gives the precise mass predictions:
\begin{equation}
M_{R, 1} \approx 9.3 \times 10^{14} \text{ GeV}, \quad M_{R, 2} \approx 1.6 \times 10^{16} \text{ GeV}, \quad M_{R, 3} \approx 1.9 \times 10^{16} \text{ GeV}
\end{equation}

\subsection{Light Neutrino Masses}

Inputting these right-handed masses and the G$_2$-derived Yukawa couplings ($Y_{\nu,3} \approx 13/11$, etc.) into the seesaw formula yields the light neutrino masses:
\begin{equation}
m_1 \approx 0.0 \text{ eV}, \quad m_2 \approx 0.0086 \text{ eV}, \quad m_3 \approx 0.050 \text{ eV}
\end{equation}

These values perfectly reproduce the observed mass-squared differences $\Delta m_{21}^2$ and $\Delta m_{31}^2$ (Section 8), confirming the validity of the geometric seesaw construction.
